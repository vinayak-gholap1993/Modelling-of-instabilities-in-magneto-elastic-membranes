\documentclass[a4paper,12pt,BCOR0mm,headsepline,draft,twoside]{scrbook}
%\documentclass[a4paper,12pt,BCOR0mm,headsepline,final,twoside]{scrbook}

\usepackage{ifpdf}
\usepackage{ifdraft}

% define the type of the thesis: Studienarbeit/Diplomarbeit
% (uncomment the appropriate type)
\newcommand{\thethesis}[0]{%
  Studienarbeit/Diplomarbeit
}

% find all usages of \field and replace the whole expression
\newcommand{\field}[1]{%
  {\itshape \{#1\}}
}

\ifoptiondraft{%
  \usepackage[firsttwo,bottomafter]{draftcopy}
}

% include some useful packages
\usepackage{scrtime}                          % gain access to time stamps
\usepackage{scrpage2}                         % headers and footers
\usepackage{makeidx}                          % support for makeidx
\usepackage{array}                            % better table support
\usepackage{multicol}                         % spanning columns
\usepackage{multirow}                         % spanning rows

% include right printer driver for graphicx
\ifpdf
  \usepackage[pdftex]{graphicx}
  \pdfcompresslevel=9
\else
  \usepackage[dvips]{graphicx}
\fi

\usepackage{subfigure}
\usepackage[ngerman,english]{babel}           % switch language
\usepackage{float}                           
\usepackage{nomencl}                          % list of abbreviations
\usepackage{algorithmic}                      % typesetting of algorithms
\usepackage[plain,chapter]{algorithm}         % typesetting of algorithms
\usepackage{stfloats}                         % used to have footnotes at bottom of the page
\usepackage[final]{listings}                  % typesetting of code listings 

% define command \missing
\newcommand{\missing}[1]{\,\,(\marginpar[\hfill!$\longrightarrow$]{$\longleftarrow$!}{\bfseries 
    Missing:}\,\emph{#1})\,\,}
\newcommand{\note}[1]{\marginpar[#1]{#1}}
\newcommand{\code}[1]{\texttt{#1}}

% environment to typeset sub-figures
\newbox\subfigbox
\makeatletter
        \newenvironment{subfloat}
                {\def\caption##1{\gdef\subcapsave{\relax##1}}%
                 \let\subcapsave\@empty
                 \setbox\subfigbox\hbox
                         \bgroup}
                  {\egroup
                 \subfigure[\subcapsave]{\box\subfigbox}}
\makeatother

% list of abbreviations
\let\abbrev\nomenclature
\renewcommand{\nomname}{List of Acronyms}
\setlength{\nomlabelwidth}{.25\hsize}
\renewcommand{\nomlabel}[1]{#1 \dotfill}
\setlength{\nomitemsep}{-\parsep}
\makeglossary
\newcommand{\markup}[1]{\textbf{#1}}

% define new style for TOC
\makeatletter
\renewcommand{\numberline}[1]{%
        \makebox[0.9cm][l]{#1}\hspace{1mm}}
\renewcommand{\l@chapter}[2]{%
        \addvspace{2ex}%
        \pagebreak[3]%
        \noindent%
        \makebox[0pt][l]{%
        \rule[-3pt]{\textwidth}{0pt}}%
        {\large\textsf{\textbf{#1}}}\hfill#2%
        \par%
        \nopagebreak%
        \addvspace{1ex}%
}
\renewcommand{\l@section}[2]{%
        \addvspace{0.5ex}%
        \noindent\hspace{1cm}%
        #1\dotfill#2%
        \par%
        \nopagebreak[2]%
}
\renewcommand{\l@subsection}[2]{%
        \addvspace{0.2ex}%
        \noindent\hspace{2cm}%
        #1\dotfill#2%
        \par%
}       
\makeatother

% define new style for index 
\makeatletter
\newcommand*{\heading}[1]{%
        \makebox[0pt][l]{%
                \rule[-3pt]{\linewidth}{0pt}}%
        \textsf{\textbf{\Large #1}}\hfill\nopagebreak\vspace{4pt}}
\renewenvironment{theindex}{%
        \setlength{\columnseprule}{0.4pt}
        \setlength{\columnsep}{2em}
        \begin{multicols}{2}[\chapter*{\indexname}]
                \parindent\z@
                \parskip\z@ \@plus .3\p@\relax
                \let\item\@idxitem}%
        {\end{multicols}\clearpage}
\makeatother

% page break
\clubpenalty = 10000
\widowpenalty = 10000

% prepare generation of index
\makeindex

% put footnotes below floats at the bottom
\fnbelowfloat

\begin{document}

\frontmatter
\begin{titlepage}
  
  \begin{center}
    
    {\Huge \bf
      \field{Title}\\
    } 
    
    \vspace*{1cm}
    \thethesis im Fach Informatik
    \vspace{3cm}
    
    {\large vorgelegt von} \\
    \vspace*{0.7cm}
    {\Large \bf \field{Your name}} \\
    \vspace*{0.7cm}
    {\large geb. \field{Date of birth} in \field{Birthplace}} 
    
    \vspace{3cm}
    
    angefertigt am 

    \vspace{1cm}
    
    {\bf 
      Department Informatik \\
      Lehrstuhl f\"ur Informatik 2\\
      Programmiersysteme \\
      Friedrich-Alexander-Universit\"at Erlangen--N\"urnberg \\
      (Prof. Dr. M. Philippsen)
      }
    
    \vspace{2cm}

    Betreuer: \field{Tutor} 
    
    \vspace{1cm}
    
    Beginn der Arbeit: \field{Date} \\
    Abgabe der Arbeit: \field{Date}
    
  \end{center}
\end{titlepage}

\clearpage{\pagestyle{empty}\cleardoublepage}

\thispagestyle{empty}
\selectlanguage{ngerman}
Ich versichere, dass ich die Arbeit ohne fremde Hilfe und ohne Benutzung
anderer als der angegebenen Quellen angefertigt habe und dass die Arbeit in
gleicher oder \"ahnlicher Form noch keiner anderen Pr\"ufungsbeh\"orde
vorgelegen hat und von dieser als Teil einer Pr\"ufungsleistung angenommen
wurde. Alle Ausf\"uhrungen, die w\"ortlich oder sinngem\"a\ss{} \"ubernommen
wurden, sind als solche gekennzeichnet.

\vspace{2cm}

Der Universit\"at Erlangen-N\"urnberg, vertreten durch die Informatik 2
(Programmiersysteme), wird f\"ur Zwecke der Forschung und Lehre ein einfaches,
kostenloses, zeitlich und \"ortlich unbeschr\"anktes Nutzungsrecht an den
Arbeitsergebnissen der \thethesis einschlie\ss{}lich etwaiger Schutzrechte und
Urheberrechte einger\"aumt.

\vspace{2cm}
Erlangen, den \field{Date}

\vspace{2cm}
\field{Your name} \hfill \ 

\vspace{0,5cm}

\clearpage{\pagestyle{empty}\cleardoublepage}
\pagenumbering{roman}

\selectlanguage{english}
\pagestyle{empty}
\begin{center}
\Huge \bf Tasks
\end{center}

\begin{enumerate}
\item[Task 1] Magneto-static magnetic scalar potential (MSP) formulation:
\begin{enumerate}
\item Literature survey on non-linear magneto-elasticity and its finite element modelling \cite{dorfmann2004, dorfmann2005, reddy_toroid, barham}
\item Devise a method of modelling the thin tubular membrane toroidal geometry using the MSP formulation 
\item Set up simulation code for the 3D and axisymmetric (2.5D) problem
\item Validate the axisymmetric formulation 
\end{enumerate}
\item[Task 2] Quasi-static finite strain elasticity problem:
\begin{enumerate}
\item Understand and implement a quasi-static finite-strain elasticity problem for pure mechanical loads and deformations \cite{Pelteret2016a, Pelteret2012}
\item Extend the application code from Task 1 for the axisymmetric formulation of the MSP problem to a coupled magneto-elastic problem known as a vector-valued problem
\item Implement a constitutive material law considering the geometric and material non-linearity of an isotropic continuum elastic body \cite{Wriggers2008}
\item Implement an iterative Newton method with (quasi-static linear) load incrementing algorithm to solve the non-linear system of equations for the vector-valued displacement solution
\item Test the developed code with mechanical test problems that exhibit finite deformations with instability/buckling characteristics
\end{enumerate}
\item[Task 3] Coupled magneto-elasticity problem:
\begin{enumerate}
\item Implement a material model for the coupled problem \cite{pelteret2016, Saxena2015}
\item Extend the application code incorporating the conclusions drawn from Task 1 and Task 2
\item Implement a segregated iterative solver from the literature \cite{Benzi2005} to solve the coupled saddle point system
\item Examine material behaviour at high mechanical and magnetic loads for material instability
\end{enumerate}
\item[Task 4] Path-following solution methods:
\begin{enumerate}
\item Literature survey of the path-following non-linear solvers \cite{Vasios, Wriggers2008, Riks1979, CRISFIELD1981}
\item Implement the Arc-Length solver considering the Crisfield method
\end{enumerate}
\end{enumerate}
\pagestyle{headings}

\pagenumbering{roman}
\include{abstract}
\clearpage{\pagestyle{plain}\cleardoublepage}

\tableofcontents
\listoffigures
\listoftables
\listofalgorithms

\mainmatter
\chapter{\textit{Introduction}}\label{chap:1}


\include{chapter2}
\include{chapter3}

\manualmark
\markleft{List of Acronyms}
\markright{List of Acronyms}
%\printglossary
\cleardoublepage

\markleft{Index}
\markright{Index}
\printindex

\automark[chapter]{section}
\bibliographystyle{plain}
\bibliography{references}

\end{document}
