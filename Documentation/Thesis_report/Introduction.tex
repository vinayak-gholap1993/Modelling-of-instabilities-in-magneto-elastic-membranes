\chapter{Introduction}
\label{chap:1}

\section{Motivation}
Magnetorheological elastomers (MREs) are soft materials which consist of micron sized iron particles dispersed in an elastomeric matrix. Magneto-active polymers (MAPs) termed as smart materials of future are capable to change their mechanical properties under the application of magnetic loads. This is due to the interaction of the ferrous magnetizable particles which gives rise to a coupled magneto-mechanical effect. In the presence of an external magnetic field, the magnetization vectors in iron particles align with the externally applied magnetic field and this causes a change in the mechanical stiffness and the macroscopic deformation of the polymer. Various polymers such as natural and synthetic rubber, hydrogels, silicon elastomers and polyurethanes have been utilized as the matrix material to synthesize MAPs \cite{Hana}. The different magnetic filler particles used are iron particles in micro- and nano-size. The proportion of these filler particles is usually between 0 and 30\% by volume of the entire mixture \cite{Saxena2015}. The finite deformations arising from the coupled interaction of the magnetic and mechanical fields is desired in many industrial applications. MAPs are synthesized and tuned with different micro-structural properties for a variety of applications such as magnetic actuators and vibrational dampers in mechatronic sensors and artificial muscles in the biomedical field. \par 
Motivated by the plethora of applications, the theory of MAPs macroscopic behaviour at continuum level proposed by \cite{tiersten1964coupled, brown1966magnetoelastic} is reviewed again by researchers. New modelling approaches were recently developed by \cite{kovetz2000electromagnetic, KANKANALA, dorfmann2004, dorfmann2005, ogden2011mechanics, Dorfmann2014}. These recent contributions form the theoretical foundation for the present work described in this thesis. In the work by \cite{dorfmann2004, dorfmann2005} it was pointed out that any one of the three magnetic vector fields, namely the magnetic field $\mathbb{H}$, the magnetic induction $\mathbb{B}$ and the magnetization $\mathbb{M}$, can be used as an independent input field to proceed with the mathematical modelling. The other two fields can be obtained through constitutive relations. Two different formulations depending on the choice of the independent field, termed as the scalar magnetic potential formulation and the vector magnetic potential formulation, were presented in \cite{dorfmann2004, dorfmann2005}. The difficulties arising in terms of numerical modelling and the restrictions on the choice of the class of constitutive material models to describe MAPs was also highlighted therein. \par
The solution of non-trivial boundary value problems through numerical modelling was the obvious next step in understanding the nature of the coupled response of these smart materials. The elastic partial differential equations (PDEs) for the material deformation and the magnetic PDEs for the magnetic field are solved using the favourable discretization approach such as Finite Element method (FEM). FE simulations of the deformations of the magneto-elastic materials have been performed considering different geometries of interest in \cite{barham, bustamante2011numerical, Vogel2014}. Due to the applications of MAPs in the critical fields such as aerospace and biomedical engineering, the investigation of the failure and stability analysis of the materials under applied loads is of vital importance. Research studies in understanding the material behaviour near critical limit points and post limit points is carried out both analytically and numerically considering an isotropic and anisotropic material micro-structure in \cite{rudykh2013stability, reddy_toroid, Reddy2018, Barham2008, Miehe2015}. In an isotropic MRE, the iron particles are distributed randomly in the matrix and the resulting macroscopic response is thus isotropic. In an anisotropic MRE, the particles are arranged in chains due to the curing process during the fabrication of the material. Different numerical experiments were carried in this direction considering the chain-like micro-structure \cite{Saxena2015} and a multiscale approach using computational homogenization in \cite{keip2016multiscale}. \par 
Of particular interest to this thesis are the recent studies carried out in \cite{reddy_toroid, Reddy2018} for isotropic MREs with axisymmetric geometries under inflating pressure loads. In case of membranes of finite thickness it is already known from the stability and buckling analysis that the elastic limit points are an important phenomenon during the free inflation of these thin membranes under finite pressure loads. At the critical limit points, the membranes start to deform significantly under a slight increase of pressure load. The material softens and due to reduced stiffness we may observe buckling or wrinkling effects in the membranes. The behaviour of the material beyond these limit points leads to multiple equilibrium states, stable or unstable or both. Hence, it is important to know a priori the limit point and the limit pressure load a membrane can withstand before a structural failure is observed. The instability behaviour of such membranes is also dependent on the constitutive material model employed to describe the magneto-elastic material response. The presence of a magnetic limit point instability, a state where both the stable and unstable equilibriums merge and cease to exist, was described in \cite{Barham2008, reddy_toroid}. \par 
In the macroscopic modelling approaches \cite{bustamante2013, pelteret2016}, the energy contributions arising from the surrounding free space around the magneto-elastic membrane are also taken into consideration. The forces arising from the Maxwell stress developed in the free space by the permeating magnetic field are significant and thus need to be taken into account. The approach of a FE mesh with truncated free space from \cite{pelteret2016} was adopted in this thesis. The magneto-elastic membrane geometry along with the truncated surrounding free space was discretized using finite element basis in order to capture a more accurate response of the deformable membrane and the free space. The free space was modelled as an elastic deformable solid with relatively low elastic stiffness compared to the magneto-elastic membrane stiffness. \par 
The research plan for this thesis is to make an attempt to model the instabilities in the finite thickness torus membranes employing the h-adaptive mesh refinement finite elements. The lack of research in the numerical modelling of axisymmetric magneto-elastic solids modelled with the surrounding free space employing FEM is the main motivation for the thesis. Development of a robust and computationally efficient multi-physics, fully coupled finite element framework for the magneto-elasticity problems employing open-source, high-performance C++ FEM library such as \texttt{deal.II} \cite{BangerthHartmannKanschat2007, dealII90} is the basic goal. The developed framework tested with various sample models and unit tests to proof-check the correct functioning may be useful as a starting point for the further research. \par

\section{Research objectives}
The research direction and the discrete objectives aimed to achieve in this thesis are precisely listed below.
\begin{itemize}
\item Magneto-static magnetic scalar potential (MSP) formulation:
\begin{enumerate}
\item Devise a method of modelling the thin tubular membrane toroidal geometry using the MSP formulation 
\item Set up simulation code for the 3D and axisymmetric (2.5D) problem
\item Validate the axisymmetric formulation 
\end{enumerate}
\item Quasi-static finite strain elasticity problem:
\begin{enumerate}
\item Understand and implement a quasi-static finite-strain elasticity problem for pure mechanical loads and deformations \cite{Pelteret2016a, Pelteret2012}
\item Extend the application code from previous task to include finite elasticity
\item Implement a constitutive material law considering the geometric and material non-linearity of an isotropic continuum elastic body \cite{Wriggers2008}
\item Implement an iterative Newton method with (quasi-static linear) load incrementing algorithm to solve for the vector-valued displacements
\item Test the developed code with mechanical test problems that exhibit finite deformations with instability/buckling characteristics
\end{enumerate}
\item Coupled magneto-elasticity problem:
\begin{enumerate}
\item Implement a material model for the coupled problem \cite{pelteret2016, Saxena2015}
\item Extend the application code incorporating the conclusions drawn from both of the previous decoupled problems
\item Implement a segregated iterative solver from the literature \cite{Benzi2005} to solve the coupled saddle point system
\item Examine material behaviour at high mechanical and magnetic loads for material instability
\end{enumerate}
\end{itemize}

\section{Outline of the thesis}
This thesis is organised into four chapters. The first chapter introduced the magneto-active polymers and presents a short summary of the research performed and undergoing in these smart materials. It motivates the need to model the unstable behaviour of inflating membranes of finite thickness and also points to its importance in practical applications in different industries. As mentioned briefly above in the research objectives, the thesis was carried out in three major milestones described in the next three chapters. In the second chapter, the theory for non-linear magneto-elasticity is laid out. A particular axisymmetric (2.5D) and a 3D geometry for the membrane and the free space is developed. Different experiments are performed to model a magnetic field closely fitting to the desired results from the existing literature. Validation of the axisymmetric formulation is carried out by comparing the results against the results of a 3D model simulation. The third chapter is dedicated to the finite deformations of the membrane along with the free space under quasi-static inflating pressure loads. Here, the membrane is modelled employing a standard Neo-Hookean constitutive material law and the surrounding free space as another non-linear complaint elastic solid. The considered solution field is the vector-valued displacement field only. Linearisation of the non-linear quantities is carried out and a non-linear Newton solver is implemented. A number of mechanical test problems are presented to proof-check the implementation at various stages of code development and to validate the behaviour of the modelling approach. Test models are also developed from the literature to study the instability behaviour under finite strain elasticity framework and to demonstrate the failure of Newton method at/near critical limit points. The fourth chapter deals with coupling of the problems from Chapter 2 and Chapter 3. Different conclusions drawn from the results of previous chapters are taken into consideration to modify the approach in order to model the instabilities. The implemented material model in Chapter 3 is extended to account for the additional energy contribution from the magnetic loads and the coupled interaction of the elastic and magnetic fields. Corresponding linearisations for the updated material model are derived and implemented to form a linear system of equations. The solution strategies of the resulting saddle point system are discussed in detail. A test model for the coupled problem to gain understanding of the behaviour of the material under combined loads is simulated. Finally, an attempt is made to initiate and model the buckling instability in the membrane near the limit points.