\chapter{Summary and outlook}
To summarize this thesis, a robust framework for coupled multi-physics problem of magneto-elasticity was developed using an open-source C++ FEM library. The framework was capable of solving the individual decoupled fields, namely the magnetic scalar potential and the vector-valued displacement fields, and also the fully coupled problem for finite deformations. The framework was tested with many simple models and unit tests to validate the implementation as well as the obtained results. Axisymmetric solids along with axisymmetric loads and boundary conditions are focussed on, in particular. \par 
In the first part of the research work, we solved a magneto-static problem involving no deformations of the domain. Here, the considered magnetic balance equations were discretized employing a scalar magnetic potential formulation. The magnetic field $\mathbb{H}$ was taken as the independent field and defined in terms of a fictitious magnetic scalar potential $\phi$. The resulting variational axisymmetric formulation was numerically solved using standard Lagrange finite elements. To validate the results obtained using the axisymmetric geometry, an analogous 3D geometry was developed for the magneto-elastic membrane and the surrounding free space. The results of the 3D simulation were compared to the axisymmetric (2.5D) simulation. An energy metric was computed to account for the total energy present in the membrane. Comparison of the energy metric values in 2.5D and 3D simulation for each h-adpative mesh refinement cycle were done. It was observed that the developed axisymmetric formulation was numerically correct under acceptable tolerance. \par 
In the second phase of research, the finite elastic deformations of the torus membrane were studied under the applied quasi-static inflating pressure load. The existing framework of magneto-statics was extended to include the non-linear elasticity problem. The magneto-elastic membrane and the surrounding free space were modelled using the hyperelastic Neo-Hookean constitutive law. Linearisation of the non-linear problem was presented using the first-order Taylor expansion about the current known solution. The vector-valued problem was discretized using vector-valued finite elements. A uniformly increasing load stepping algorithm was implemented and the solution of the non-linear system was carried out employing the Newton-Raphson iterative method. Since the free space was modelled as a very complaint elastic solid of relatively low elastic stiffness, a parametric study of the free space shear modulus $\mu$ was performed. An appropriate elastic stiffness of the free space material was found in order to have no obstructive/reactive force from the free space on the free inflation of the membrane. The instability problem was examined for finite-strain elasticity test models from the literature modelled without any free space. The failure of the Newton method to capture the unstable deformation in the Crisfield beam model was presented. Thus, the need of the path-following solution method was highlighted for the further instability studies in coupled problem. \par 
In the final part, coupling between the magneto-static and the finite-strain elasticity problem was implemented. The material model for the coupled magneto-elastic membrane was presented and the constitutive relations and tangents were derived. Conclusions were drawn from the results of the experiment for the permanent magnet region with the applied potential in the magneto-static problem and the application of the magnetic loads for the coupled problem was altered appropriately. The solution of the resulting saddle point system was carried out by the Schur complement reduction. To examine the unstable buckling deformations of the membrane, a thin membrane of reduced thickness was considered. An attempt to model the instabilities at/near limit point loads was presented. The reason for the observed numerical failure in the free space elements at the unstable deformation of the magneto-elastic membrane was highlighted.

\section*{Possible extensions}
As observed in the final results for the buckling instabilities in the membrane for the coupled problem, one immediate possible extension to circumvent the numerical failure, arising due to non-invertible deformation maps in the free space region neighbouring the buckling membrane, is employing a staggered modelling approach. The problem will be split in two sub-problems. The first sub-problem will be to solve for the finite deformations of the elastic membrane neglecting the degrees of freedom in the free space. In the second sub-problem the obtained deformations of the membrane in the earlier sub-problem will be used as the inhomogeneous Dirichlet boundary conditions. A boundary value problem will be solved for the deformations of the free space elements with resulting invertible deformation maps. The approach is termed the ``moving-mesh update'' in the literature for the fluid mechanics and fluid-structure interaction problems. \par 
Another avenue of extension will be to get the implemented the path-following Arc-Length solver to function correctly. The current Newton method would fail past the critical limit points due to the negative elastic tangents and the lack of the Newton solver to dynamically adapt the load parameter value. \par 
Certainly one could also employ more complex constitutive models such as the Mooney-Rivlin and Arruda-Boyce material model to study their instabilities in depth. Employing an incompressible magneto-elastic membrane for such complex modelling involving complex geometry and external load applications is another open-ended research direction to explore.